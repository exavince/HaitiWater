\documentclass{EPL-master-thesis-covers-FR}

\title{HaïtiWater 2.0}
\subtitle{Evolution de l'application HaïtiWater vers une application entière  fonctionnelle hors-ligne}

\author{Vincent \textsc{Gradzielewski}}% Handcrafted third author :D

\degreetitle{Master [120] en sciences informatiques}

\supervisor{Kim \textsc{Mens}}
\secondsupervisor{Sandra \textsc{Soares-Frazão}}

\readerone{}
\readertwo{}

\years{2020-2021}

\usepackage{hyperref}
\usepackage{cite}
\usepackage{float}
\usepackage{multirow}
\usepackage{multicol}
\usepackage[final]{pdfpages}
\usepackage{booktabs}
\usepackage{multirow}
\usepackage{graphicx}
\usepackage[toc]{multitoc}

\usepackage{amssymb}% http://ctan.org/pkg/amssymb
\usepackage{pifont}% http://ctan.org/pkg/pifont
\newcommand{\cmark}{\ding{51}}% pour les checkmark
\newcommand{\xmark}{\ding{55}}%

\frenchbsetup{StandardLists=true} % Resolves conflict between babel and enumitem
\usepackage{enumitem} % better formating of lists

\usepackage[style=long,nonumberlist,toc,xindy,acronym,nomain]{glossaries}
\makenoidxglossaries
\input{../documents/requirements/glossary.gloss}

\begin{document}

	\maketitle
	\tableofcontents

	\setlength{\parskip}{1.5em plus1em minus1em}

	% Total des pages : entre 49 et 75 d'après nos estimations.

	\chapter*{Résumé}
	\addcontentsline{toc}{chapter}{Résumé}
	
		Ce travail de fin d'études a été réalisé dans le cadre de mon Master en Sciences Informatiques à l'École Polytechnique de Louvain-la-neuve durant l'année académique 2020-2021.
		
		Dans ce mémoire je vais présenter mon travail qui consistait à reprendre l'application HaïtiWater développée précedemment par Adrien Hallet, Céline Deknop et Sebastien Strebelle afin de la faire évoluer vers une application web qui serait entièrement utilisable hors-ligne.
		 
		Cette application a pour but "La gestion du réseau de distribution d'eau potable en Haïti". Je commencerai par une brève introduction sur le contexte Haitien et sur les raisons pour lesquels l'évolution de cette application était nécessaire. Ensuite je présenterai mes idées et réalisations en justifiant à chaque fois les différents choix d'implémentation qui on été fait et pourquoi j'ai utilisé certaines technologies plutot que d'autres. Je présenterai ensuite la validation de l'application et les feedbacks que j'ai reçu des utilisateurs. Puis je concluerai par une liste des améliorations possibles.

		Tout le travail réaliser est disponible ici :

\begin{itemize}
	\item Github : \url{https://github.com/exavince/HaitiWater}
	\item Par l'UCL : \url{https://haitiwater.sipr.ucl.ac.be}
	\item En Haïti : \url{unknown}
\end{itemize}		

		Si vous désirez tester l'application, il suffit de vous connecter sur un des liens cité précedemment de vous connecter à l'aide de l'uilisateur \emph{Protos} dont le mot de passe est également \emph{Protos}. Cependant je vous demanderai de ne pas modifier les données présentes car celle-ci ne sont pas des données fictives et on été entrée dans le but de tester l'application en Haïti.
		
		 le nom d'utilisateur principal ainsi que son mot de passe est \emph{Protos}. Nous vous demandons cependant de ne pas modifier les données présentes, car elles sont réelles et ont été entrées dans le but de tester l'application sur place.

	\chapter*{Remerciements}
	\addcontentsline{toc}{chapter}{Remerciements}

		

	%\printnoidxglossary[title=Glossaire, toctitle=Glossaire]
	%\glsaddall

	\chapter{Introduction}

		
		\subsection*{Contexte}
		
			Ce mémoire appartient à un projet de développement financé par ARES-CCD avec quelques partenaires tels que Protos\footnote{\href{https://www.protos.ngo/fr/}{www.protos.ngo}}, l'UCL et l'UEH. 
			
			Protos est une ONG qui vise à améliorer l'accès à l'eau potable dans plusieurs pays du monde afin de les aider à se développer. 
			
			Suite à de nombreuses crises politiques et catastrophes naturelles qui ont détruit beaucoup d'infrastructure locale, l'accès à l'eau potables est devenu difficile en Haïti. De plus, des incertitudes politiques entravent la reconstruction de ces installations et les populations ne sont pas toujours aidées par les services publics pour assurer la distribution de l'eau. C'est pour cette rainson que l'ONG Protos est active dans le pays depuis quelques années et à permis aux anciens mémorants de créer l'application HaïtiWater.
						
			Il y a quelques années Protos est entré en contact avec l'UCL afin de réaliser un système logiciel pilote pour la gestion de la distribution d'eau potable en zone rurale.
			
			 En effet, aucune gestion centralisée organisée par l'Etat n'existe pour ces zones, éloignées des grandes agglomérations. Des réseaux existent, constitués de points de prélèvement d'eau, de conduites de distribution d'eau et de fontaines situées dans les villages, mais la gestion publique de ceux-ci n'est pas opérationnelle. 

			L'application créée précedemment propose un appui à ces organismes locaux afin de mieux organiser cette distribution. 
			%Explication de l'application des différents modules existants

			

		\subsection*{Problématiques}

			% Expliquer pourquoi l'application a besoin d'évoluer 

		\subsection*{Motivation}

			%Explqiuer pourquoi avoir choisis ce mémoire
			

		\subsection*{Objectifs}

			%Expliquer le but final du mémoire au niveau de l'usage de l'applcition en Haiti
			
		\subsection*{Approche}

			%Expliquer les différentes étapes par lesquelles je suis passé pour réaliser ce mémoire (étude des différentes tech, développement, validation, ...)
			
		\subsection*{Contribution}

			%Qu'est ce qu'a apporté le travail réalisé
			%Qu'elle est la plus value ?

		\subsection*{Plan}

			%Explication dans le temps des différentes phases de travail 

	\chapter{Contexte}


		\section{La gestion de l'eau en Haïti}
			\label{sec:situation}

				%Brève explication sur le contexte Haïtien


		\section{Introduction à l'application créer précedemment}

				

		\section{Problèmes réseaux}

				

	\chapter{Organisation}
		

		%Brève intro à l'organisation

		\section{Approche de travail}

			%Etude du problème
			%Prioritisation des taches

			\subsection*{Planification}
				\label{sec:planification}

				%Planification des taches sur toutes l'année (reprendre planning développé pour le Q2)
		\section{Méthodologie}

			%Mise en place d'une méthode pour garantir l'avancement du projet

			\subsection*{Agile}
			
				%Pourquoi une méthode agile non fixe plutot que waterfall

				%Dev par fonctionnalité
				%Itération 
				

			\subsection*{Phases du mémoire}

			

	\chapter{Analyse des besoins}
		\label{sec:analyse_besoins}

		% Phase essentiele blabla
		% Retouver les schémas créé début d'année

		\section{Besoins fonctionnels}

			
			%éfinition

			\subsection*{Gestion des données}
				\label{sec:gest_donnee}



			\subsection*{Affichage et accès}

			

		\section{Besoins non-fonctionnels}

			%Définition

			\subsection*{Sécurité des données}


			\subsection*{Multi-plateforme}
				

			\subsection*{Technologies simples et populaires}


		\section{Structure modulaire}
			\label{sec:modules}

			%Expliquer que mon travail conserve la structure modulaire
			%Fin du module offline
			%Lien vers ancien mémoire
			%Ajout du module "A modifier"

			

		\section{Structure des données hors-ligne}
			\label{sec:data}

			%Expliquer la différence entre données online/offline
			%Introduire ma solution

	\chapter{Implémentation}

		\section{Description de l'app de base}
			%Expliquer l'implémentation de l'app existante
				%Django
				%PostGres
				%...
		
		\section{Choix technologiques}
			\label{sec:choix_tech}
			
			\subsection*{Progressive Web-App}
				%Expliquer les différentes types d'application hors-ligne étudiée
				
			\subsection{Stratégie de synchronisation des pages}			
			
			\subsection{Stratégie de synchronisation de la DB}
			
			
			\subsection{Gestion des changements d'utilisateurs}		
			
			
			\subsection{Sytème de notification}	
						

			\subsection*{Service-worker}

		
			\subsection*{IndexedDB}

					

			\subsection*{Dexie.js}
				

			\subsection*{DataTables}

			

			\subsection*{Chart.JS}
				
				

		\section{La hiérarchie dans l'application}
			%Epliquer qu'il y a eu ou pas des changements dans le fonctionnement de l'application au niveau des modules et de la hiérachie


			\subsection*{Structure}

			

			\subsection*{Permissions}

			

		\section{Interface utilisateur}
			%Expliquer les changements faits par rapport à l'interface de base
			%Ajout de couleurs, notifications, données non-sync, ...

		\section{Client}

			%Introduction

			\subsection*{Mise en cache}
				\label{sec:cache_client}

			
			\subsection*{Gestion des données}
				Gabarits, modularité et réactivité

			\subsection*{Push des données hors-ligne}
				\label{sec:service_worker}
				
				

			

		\section{Serveur}
			\label{sec:serveur}

		

			\subsection*{Requêtes}

			

			\subsection*{Détails des requêtes API}
				\label{sec:api}

			

	\chapter{Validation}

		%Introduction

		\section{Vérifications automatiques}

			\subsection*{Tests unitaires}

			

		\section{Vérifications utilisateurs réels}


			\subsection*{Méthodologie}

				

			\subsection*{Résultats obtenus}

				

			\subsection*{Modifications apportées}

		

	\chapter{Améliorations futures}


		\section{Suite du projet}
			\label{ref:suite_projet}

		

		\section{Défis rencontrés}

			

		\section{Propositions}

			
	\chapter{Conclusion}

		

		\section{Métriques}
		

	\bibliography{bibliography}{}
	\bibliographystyle{plain}
	\addcontentsline{toc}{chapter}{Bibliographie}
		

	

	\setlength{\parskip}{0em}
	\backcoverpage

\end{document}
